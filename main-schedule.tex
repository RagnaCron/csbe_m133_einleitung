%%
%% Copyright (C) 2020 by Manuel Werder
%%
%% This work may be distributed and/or modified under the
%% conditions of the LaTeX Project Public License, either version 1.3
%% of this license or (at your option) any later version.
%% The latest version of this license is in
%%
%%     http://www.latex-project.org/lppl.txt
%%
%! Date = 22.09.20


% Preamble
\documentclass[landscape,11pt,table]{manual}

\usepackage{longtable}
\usepackage{multirow}

% Document
\begin{document}
    \pagestyle{plain}
    \fancyhf{}
    \lhead{Modul 133}
    \chead{Webapplikation mit Session-Handling realisieren}
    \rhead{\gitAuthorDate}
    \lfoot{CSBE \copyright\space 2020 Manuel Werder, Bruno Trachsel}
    \cfoot{\gitStyler}
    \rfoot{Seite \thepage}

    \begin{table}[h!]
        \centering
        \begin{tabular}{ p{0.2\textwidth} p{\textwidth} }
        {{\raggedright}
        {\raggedright}}
            \textbf{Klasse:} & INF 19P \\ [1em]
            \textbf{LBV:}  & \textbf{LBV Modul 133 \textemdash\space 2 Elemente \textemdash\space Schriftliche Einzelprüfung, Projektarbeit mit Dokumentation} \\
            \textbf{Fachmodul:} & 133 Webapplikation mit Session-Handling realisieren \\
            \textbf{Kompetenzziel:} & Webapplikation gemäss Vorgabe mit einer Programmiersprache realisieren und testen. \\
            \textbf{Kompetenzfeld:} & Web Engineering \\
            \textbf{Fachliteratur:} & Modul Scripte, 100, 101, 105, 133, 403, 404, (226A, 226B)\newline
            \href{http://openbook.rheinwerk-verlag.de/webdesign/}{Webdesign Das Handbuch zur Webgestaltung}\newline
            ISBN 978-3-8362-4402-2, 2. Auflage, 2017\newline
            \href{http://openbook.rheinwerk-verlag.de/oop/}{Objektorientierte Programmierung Das umfassende Handbuch}\newline
            ISBN 978-3-8362-1401-8, 2., aktualisierte und erweiterte\newline
            \href{http://openbook.rheinwerk-verlag.de/it_handbuch/}{IT-Handbuch für Fachinformatiker}\newline
            ISBN 978-3-8362-2234-1, 6. aktualisierte und überarbeitete Auflage\newline
            \href{http://openbook.rheinwerk-verlag.de/linux/}{Linux Das umfassende Handbuch}\newline
            ISBN 978-3-8362-1822-1\\
        \end{tabular}\label{tab:table1}
    \end{table}

    \begin{tabularx}{\textwidth} {
    | >{\hsize=.1\textwidth\raggedright\arraybackslash}X
    | >{\raggedright\arraybackslash}X
    | >{\raggedright\arraybackslash}X
    | >{\raggedright\arraybackslash}X
    | >{\raggedright\arraybackslash}X | }
        \hline
        \multicolumn{5}{|c|}{\cellcolor{lightgray}\textbf{\large Übersicht Unterrichtsprogramm}} \\
        \hline
        \hline
        \rowcolor{lightgray}
        \textbf{Tag \& Datum} &
        \textbf{Inhalt} &
        \textbf{Lehrmittel} &
        \textbf{Lernziele} &
        \textbf{Selbständiges Lernen / Auftrag / Übungen} \\
        \hline
        1, 11.08.20 &
        Vorbereitung für Modul 133,
        Modulidentifikation lesen und erarbeiten des Inhaltes 00\textunderscore EinleitungModul133&
        Modulidentifikation.pdf, 00\textunderscore EinleitungModul133.pdf &
        Verstehen die Lernziele des Moduls,
        haben eine VM neu aufgesetzt und ihre arbeitsumgebung eingerichtet.&
        Selbständiges Arbeiten \\
        \hline
        2, 12.08.20 &
        Vorbereitung für Modul 133,
        00\textunderscore EinleitungModul133 fertigstellen
        und Lerneinheit AB133-01 erarbeiten &
        00\textunderscore EinleitungModul133.pdf,
        Lerneinheit AB133-01.pdf &
        Lerneinheiten erarbeiten,
        Grundlagen Ruby verstehen,
        Ruby on Rails erste versuche abgeschlossen &
        Selbständiges Arbeiten \\
        \hline
        3, 26.08.20 &
        AB133-02 und
        AB133-03 erarbeiten,
        Besprechung für die Leistungsbeurteilung 2 (Documentation und Projekt) &
        AB133-02.pdf,
        AB133-03.pdf &
        Statische Seite mit Rails erstellen, editieren und mit Twitter-
        Bootstrap designen gemäss AB133-02,
        MVC – Model View Controller: Model (Datenbank) erstellen und in
        View anzeigen gemäss AB133-03 &
        Begleiteter Unterricht,
        Selbständiges Arbeiten \\
        \hline
    \end{tabularx}

    \newpage

    \begin{tabularx}{\textwidth} {
    | >{\hsize=.1\textwidth\raggedright\arraybackslash}X
    | >{\raggedright\arraybackslash}X
    | >{\raggedright\arraybackslash}X
    | >{\raggedright\arraybackslash}X
    | >{\raggedright\arraybackslash}X | }
        \hline
        \rowcolor{lightgray}
        \textbf{Tag \& Datum} &
        \textbf{Inhalt} &
        \textbf{Lehrmittel} &
        \textbf{Lernziele} &
        \textbf{Selbständiges Lernen / Auftrag / Übungen} \\
        \hline
        4, 01.09.20 &
        AB133-04 und
        AB133-05 erarbeiten,
        Besprechung der Mockups gemäss LB Vorgabe &
        AB133-04.pdf,
        AB133-05.pdf &
        MVC: Model, Controller. list-, show-, new-, create-, edit-, update-,
        delete-Methoden in Rails realisieren gemäss AB133-04,
        Model, Controller, list, show, new, edit, show\textunderscore subject Views in
        Rails realisieren gemäss AB133-04 &
        Begleiteter Unterricht,
        Selbständiges Arbeiten \\
        \hline
        5, 09.09.20 &
        AB133-06 erarbeiten &
        AB133-06.pdf &
        Session Handling gemäss AB133-06 &
        Begleiteter Unterricht,
        Selbständiges Arbeiten \\
        \hline
        6, 14.10.20 &
        LB 1 Schriftliche Einzelprüfung,
        Arbeiten an der LB 2 Abgabe um 23:59 &
        \- &
        \- &
        Begleiteter Unterricht,
        Selbständiges Arbeiten \\
        \hline
    \end{tabularx}

\end{document}