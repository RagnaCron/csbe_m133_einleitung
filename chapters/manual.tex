%%
%% Copyright (C) 2020 by Manuel Werder
%%
%% This work may be distributed and/or modified under the
%% conditions of the LaTeX Project Public License, either version 1.3
%% of this license or (at your option) any later version.
%% The latest version of this license is in
%%
%%     http://www.latex-project.org/lppl.txt
%%
\mode*

\section{Einleitung}
\label{sec:einleitung}
\mode<article>{\setcounter{page}{1}}

\begin{frame}[fragile]
    \mode<article>{}
    Geschätzte Inf19P, in diesem Modul (133) soll eine Web-Applikation mit Session-Handling realisiert werden, dies soll
    mit einem Framework bewerkstelligt werden.
    Dazu stehen uns zwei Möglichkeiten zur Verfügung, in den Modulunterlagen die vorliegen, können Sie mit Ruby on Rails arbeiten
    oder Sie wählen Python mit Django zusammen.
    Sie sollen in diesem Modul ein Projekt mit Dokumentation erarbeiten, dies werden Sie in Einzelarbeit machen.
    \mode<article>{\vSpaceStyle{}}
    Bevor Sie nun aber gleich anfang sich Gedanken zu einem Projekt zu machen, sollten Sie zu erste sicherstellen das
    Sie eine geeignete Arbeitsumgebung haben\ldots\newline Sie werden zum Projekt zu einem späteren Zeitpunkt noch die LB Vorgaben erhalten.
\end{frame}


\subsection{Modulidentifikation}
\label{subsec:modulid}
Die Handlungsziele gemäss Modulidentifikation:
\begin{frame}[fragile]
    \frametitle<presentation>{Modulidentifikation}
    \begin{enumerate}
        \item Vorgabe analysieren, Funktionalität entwerfen und
        Realisierungskonzept festlegen.
        \item Spezifische Funktionalität einer Web-Applikation mit Session-Handling,
        Authentifizierung und Formularüberprüfungen realisieren.
        \item Web-Applikation mit einer Programmiersprache unter
        Berücksichtigung sicherheitsrelevanter Anforderungen
        programmieren.
        \item Web-Applikation gemäss Testplan auf Funktionalität und
        Sicherheit überprüfen, Testergebnisse festhalten und
        allenfalls erforderliche Korrekturen vornehmen.
    \end{enumerate}
\end{frame}


\subsection{Interessante Links}\label{subsec:projekt}
\begin{frame}[fragile]
    Hier finden Sie eine Auflistung der relevanten Links die Sie benötigen für Ihren ersten Arbeitsauftrag.
\end{frame}


\begin{frame}[fragile]
    \frametitle<presentation>{Software, Tools und Services}
    \begin{table}[h!]
        \centering
        \begin{tabularx}{0.8\textwidth} {
        | >{\raggedright\arraybackslash}X
        | >{\raggedright\arraybackslash}X | }
            \hline
            IDE RubyMine & \href{https://www.jetbrains.com/de-de/ruby/}{RubyMine} \\
            \hline
            IDE PyCharm (Professional) & \href{https://www.jetbrains.com/de-de/pycharm/}{PyCharm} \\
            \hline
            Git & \href{https://git-scm.com/}{Git} \\
            \hline
            GitHub & \href{https://github.com/}{GitHub} \\
            \hline
            Ruby & \href{https://www.ruby-lang.org/de/}{Ruby} \\
            \hline
            Ruby on Rails & \href{https://guides.rubyonrails.org/index.html}{Ruby on Rails} \\
            \hline
            Python & \href{https://www.python.org/}{Python} \\
            \hline
            Django & \href{https://www.djangoproject.com/}{Django} \\
            \hline
        \end{tabularx}
        \caption{Software, Tools und Services}
        \label{tab:2}
    \end{table}
\end{frame}

%\figureStyle{width=0.25\textwidth}{GameScreen}{Game Screen}{fig:GameScreen}



\section{Selbstständigkeitserklärung}\label{sec:ssk}
\begin{frame}[fragile]
    \frametitle<presentation>{Selbstständigkeitserklärung}
    Hiermit erkläre ich, dass dieses Dokument selbstständig verfasst wurde, dass alle Angaben
    überprüft und entsprechend korrekt referenziert sind.
    \mode<article>{\vSpaceStyle{}}
    \gitAuthorDate~\gitAuthorName
\end{frame}


