%%
%% Copyright (C) 2020 by Manuel Werder
%%
%% This work may be distributed and/or modified under the
%% conditions of the LaTeX Project Public License, either version 1.3
%% of this license or (at your option) any later version.
%% The latest version of this license is in
%%
%%     http://www.latex-project.org/lppl.txt
%%
%! Date = 13.05.20

\secStyle{Einleitung}
\label{einleitung}
\setcounter{page}{1}
\normalsize
some text...
\vSpaceStyle{}
some text

\subSecStyle{Vorgaben}
Diese werden wie folgt aufgelistet:

\begin{itemize}
    \item one
\end{itemize}


\subSecStyle{Modulidentifikation}
Die Handlungsziele gemäss Modulidentifikation:

\begin{enumerate}
    \item one
\end{enumerate}


\subSecStyle{Projektübersicht}
Es werden hier die wichtigsten Projekttools die wie Beteiligten aufgezeigt.

\begin{table}[h!]
    \centering
    \begin{tabularx}{0.8\textwidth} {
    | >{\raggedright\arraybackslash}X
    | >{\raggedright\arraybackslash}X | }
        \hline
        Student und Projektleiter & Manuel Werder \\
        \hline
%        Dozent & \sffamily   \\
%        \hline
    \end{tabularx}
    \caption{Projektbeteiligte}
    \label{tab:1}
\end{table}

\begin{table}[h!]
    \centering
    \begin{tabularx}{0.8\textwidth} {
    | >{\raggedright\arraybackslash}X
    | >{\raggedright\arraybackslash}X | }
        \hline
        \LaTeX & \href{https://www.overleaf.com}{Overleaf} \\
        \hline
    \end{tabularx}
    \caption{Tools, Software und Services}
    \label{tab:2}
\end{table}



\secStyle{Konzepte}\label{konzepte}
%\figureStyle{width=0.25\textwidth}{GameScreen}{Game Screen}{fig:GameScreen}



\secStyle{Selbstständigkeitserklärung}
Hiermit erkläre ich, dass dieses Dokument selbstständig verfasst wurde, dass alle Angaben korrekt
und überprüft sind und entsprechend korrekt referenziert wird.
\vSpaceStyle{}
\gitAuthorDate~\gitAuthorName
