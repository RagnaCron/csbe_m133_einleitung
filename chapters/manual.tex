%%
%% Copyright (C) 2020 by Manuel Werder
%%
%% This work may be distributed and/or modified under the
%% conditions of the LaTeX Project Public License, either version 1.3
%% of this license or (at your option) any later version.
%% The latest version of this license is in
%%
%%     http://www.latex-project.org/lppl.txt
%%
\mode*


\section{Einleitung}
\label{sec:einleitung}
\mode<article>{\setcounter{page}{1}}

\begin{frame}[fragile]
    \mode<article>{}
    Geschätzte Inf19P, in diesem Modul (133) soll eine Web-Applikation mit Session-Handling realisiert werden, dies soll
    mit einem Framework bewerkstelligt werden.
    Dazu stehen uns zwei Möglichkeiten zur Verfügung, in den Modulunterlagen die vorliegen, können Sie mit Ruby on Rails arbeiten
    oder Sie wählen Python mit Django zusammen.
    Sie sollen in diesem Modul ein Projekt mit Dokumentation erarbeiten, dies werden Sie in Einzelarbeit machen.
    \mode<article>{\vSpaceStyle{}}
    Bevor Sie nun aber gleich anfang sich Gedanken zu einem Projekt zu machen, sollten Sie zu erste sicherstellen das
    Sie eine geeignete Arbeitsumgebung haben\ldots\newline Sie werden zum Projekt zu einem späteren Zeitpunkt noch die LB Vorgaben erhalten.
\end{frame}

\subsection{Modulidentifikation}
\label{subsec:modulid}
Die Handlungsziele gemäss Modulidentifikation:
\begin{frame}[fragile]
    \frametitle<presentation>{Modulidentifikation}
    \begin{enumerate}
        \item Vorgabe analysieren, Funktionalität entwerfen und
        Realisierungskonzept festlegen.
        \item Spezifische Funktionalität einer Web-Applikation mit Session-Handling,
        Authentifizierung und Formularüberprüfungen realisieren.
        \item Web-Applikation mit einer Programmiersprache unter
        Berücksichtigung sicherheitsrelevanter Anforderungen
        programmieren.
        \item Web-Applikation gemäss Testplan auf Funktionalität und
        Sicherheit überprüfen, Testergebnisse festhalten und
        allenfalls erforderliche Korrekturen vornehmen.
    \end{enumerate}
\end{frame}

\subsection{Interessante Links}\label{subsec:links}
\begin{frame}[fragile]
    Hier finden Sie eine Auflistung der relevanten Links die Sie benötigen für Ihren ersten Arbeitsauftrag.
\end{frame}


\begin{frame}[fragile]
    \frametitle<presentation>{Software, Tools und Services}
    \begin{table}[h!]
        \centering
        \begin{tabularx}{0.8\textwidth} {
        | >{\raggedright\arraybackslash}X
        | >{\raggedright\arraybackslash}X | }
            \hline
            Ubuntu 20.04               & \href{https://ubuntu.com/}{Ubuntu}                              \\
            \hline
            IDE RubyMine               & \href{https://www.jetbrains.com/de-de/ruby/}{RubyMine}          \\
            \hline
            IDE PyCharm (Professional) & \href{https://www.jetbrains.com/de-de/pycharm/}{PyCharm}        \\
            \hline
            Git                        & \href{https://git-scm.com/}{Git}                                \\
            \hline
            GitHub                     & \href{https://github.com/}{GitHub}                              \\
            \hline
            Ruby                       & \href{https://www.ruby-lang.org/de/}{Ruby}                      \\
            \hline
            Ruby on Rails              & \href{https://guides.rubyonrails.org/index.html}{Ruby on Rails} \\
            \hline
            Python                     & \href{https://www.python.org/}{Python}                          \\
            \hline
            Django                     & \href{https://www.djangoproject.com/}{Django}                   \\
            \hline
        \end{tabularx}
        \caption{Software, Tools und Services}
        \label{tab:2}
    \end{table}
\end{frame}

%\figureStyle{width=0.25\textwidth}{GameScreen}{Game Screen}{fig:GameScreen}


\section{Saubere Arbeitsumgebung}\label{sec:sauber}
\begin{frame}[fragile]
    Damit alle über die gleichen Voraussetzungen für dieses Modul verfügen, wird hier eine kleine Anleitung zum
    Besten gegeben, wie das Sie eine neue Linux Ubuntu 20.04 VM installieren und dann dem entsprechend konfigurieren
    mit der Software die Sie benötigen werden für das Projekt.
\end{frame}

\subsection{VM installieren}\label{subsec:vminstallieren}
\begin{frame}[fragile]
    Für die neue Installation von einem Ubuntu 20.04 benötigen Sie zu erste Mal das ISO, dieses können Sie
    unter dem Ordner "Software4CsBe dann ISO" holen.
    Während Sie auf den Download warten, beginnen Sie mit der Konfiguration der neuen VM.
\end{frame}

\begin{frame}[fragile]
    \mode<article>{\vSpaceStyle{}}
    \begin{enumerate}
        \item Öffnen Sie VMWare Player und erstellen Sie eine neue VM.
        \figureStyle{width=0.50\textwidth}{01_CreateVM}{Neue VM erstellen}{fig:createVM}
        \item Sie wollen das ISO zu einem späteren Zeitpunkt übergeben, wählen Sie die entsprechende Option.
        \figureStyle{width=0.60\textwidth}{02_iso_later}{Installiere das OS später}{fig:isolater}
        \item Wählen Sie nun Betriebssystem aus das Sie später installieren wollen.
        \figureStyle{width=0.60\textwidth}{03_guest_os}{Betriebssystem wählen}{fig:guestOS} \newpage
        \item Geben Sie der VM einen Namen, z.B. "vmUbuntu20.04\textunderscore M133", für den Speicherort Ihrer VM wählen Sie die eigene
        SSD aus (Somit Sie selbstverständlich auch von zu Hause aus arbeiten können).
        \figureStyle{width=0.70\textwidth}{04_vm_name}{VM Name \& Speicherort}{fig:saveVM}
        \item Geben Sie Ihrer VM eine Disk grösse von ca. 60 GB und speichern Sie diese als single file.
        \figureStyle{width=0.70\textwidth}{05_disk_size}{Disk grösse \& Single File}{fig:diskSize}
        \item Schliessen Sie nun die Konfiguration mit Finish ab.
        \item Nun sollte das ISO in der Zwischenzeit heruntergeladen Sein.
        Übergeben Sie dieses und speichern dann.
        \item Starten Sie die VM und stellen Sie die Installation des Betriebssystems fertig.
        \item Wählen Sie entweder Deutsch oder Englisch für die Sprache aus (Wenn Sie eine andere Sprache wählen kann ich ihnen keine Hilfe geben bei ihrer VM, falls sie probleme hat.)
        Klicken Sie auf Install Ubuntu.
        \item Suchen Sie das richtige Tastaturlayout aus. (Dazu können Sie auch auf Tastaturbelegung automatisch erkennen klicken) \newpage
        \item Wählen Sie nun folgende Konfiguration aus.
        \figureStyle{width=0.80\textwidth}{06_config}{Minimale Konfiguration}{fig:configVM}
        \item Kicken Sie nun auf Installieren.
        \item Benennen Sie den Benutzer und Computername wie Sie möchten (Stellen Sie aber sicher, dass niemand Anderes den gleichen Computernamen hat).
        Nehmen Sie sich ein einfaches Passwort, dass Sie sich merken können.
        \item Sie sind nun im Besitz einer VM, mit der man nun entsprechende Software für das Modul installieren kann.
        \item Starten Sie die VM neu nach der Installation und führen Sie ein Softwareupdate durch.
    \end{enumerate}
\end{frame}


\section{Alles was ein Softwareentwickler braucht}\label{sec:software}
\begin{frame}[fragile]
    In diesem Kapitel wird sämtliche Software installiert die Sie in diesem Modul benötigen.
\end{frame}

\subsection{GitHub}\label{subsec:github}
\begin{frame}[fragile]
    Falls Sie noch nicht über einen GitHub Account verfügen, erstellen Sie nun einen, Sie werden diesen in diesem Modul
    zwingend Benötigen.
    Folgen Sie diesem Link, falls Sie keinen GitHub Account haben: \href{https://github.com/join}{Join GitHub}.
    \mode<article>{\vSpaceStyle{}}
    Ansonsten gehen Sie weiter zum nächsten Kapitel.
\end{frame}
\newpage

\subsection{Git}\label{subsec:git}
\begin{frame}[fragile]
    Auf Ihrer neu erstellten Ubuntu VM wird nun Git installiert.
    \begin{enumerate}
        \item Öffnen Sie das Terminal (Tastenkombination: \textbf{ctrl + alt + t})
        \item Geben Sie folgenden Befehl als Erstes ein: \textbf{sudo apt-get install update}
        \item Geben Sie folgenden Befehl ein um Git zu installieren: \textbf{sudo apt-get -y install git}
        \item Somit haben Sie Git erfolgreich installiert.
    \end{enumerate}
    In den Modulunterlagen im Ordner 02\textunderscore Scrips finden Sie ein PDF namens progit.pdf lesen Sie nun
    Kapitel 1 - 3 durch.
    Sie können dies auch Online tun: \href{https://git-scm.com/book/en/v2}{Git Pro Book}
    \mode<article>{\vSpaceStyle{}}
    Es wird nun die basis Konfiguration für Git vorgenommen, Sie können diese in Kapitel 1 vom Git Pro pdf nachlesen.
    \begin{enumerate}
        \item Öffnen Sie das Terminal (Tastenkombination: \textbf{ctrl + alt + t})
        \item Geben Sie folgenden Befehl ein um Ihren persönlichen Namen zu setzen: \newline\textbf{git config \texttt{--}global user.name "John Doe"}
        \item Geben Sie folgenden Befehl ein um Ihre E-Mail-Adresse zu setzen: \newline\textbf{git config \texttt{--}global user.email johndoe@example.com}
    \end{enumerate}
\end{frame}

\subsection{JetBrains Produkte}\label{subsec:jetbrains}
\begin{frame}[fragile]
    Für die Entwicklung Ihres Projektes benötigen Sie eine von zwei Entwicklungsumgebungen.
    Beide stammen aus dem Hause JetBrains und bietet Ihnen, die entsprechende funktionalität, die eine IDE haben sollte.
    \mode<article>{\vSpaceStyle{}}
    Programmierunterstützende Tools nennen sich IDE / (Integrated Development Environment) /
    Integrierte Entwicklungsumgebung.
    Diese Entwicklungsumgebungen verfügen meist über folgende Features:
    \begin{itemize}
        \item Autovervollständigung
        \item IntelliSense (Vervollständigung und Lernfähige Bibliothek)
        \item Integration in Versionisierungssysteme
        \item Hilfsmittel zur gemeinsamen Zusammenarbeit
        \item Integration in das Datenbankschema (Visualisierte Darstellung der Datenbank)
        \item Ggf. Grafische Editoren
        \item Ggf. Automatische Code-Generierung
        \item Uvm.
    \end{itemize}
    \mode<article>{\vSpaceStyle{}}
    Falls Sie bereits durch ein früheres Modul über einen JetBrains Account verfügen, gehen sie zum Ende dieses
    Unterkapitels und installieren Sie beide IDE's.
\end{frame}

\subsubsection{JetBrains Account erstellen}\label{subsubsec:jetbrainsAccount}
\begin{frame}[fragile]
    \begin{enumerate}
        \item Registrieren Sie sich auf JetBrains mit ihrer Schul-E-Mail-Adresse.
        \href{https://account.jetbrains.com/login}{JetBrains Register}
        \figureStyle{width=0.60\textwidth}{07_not_registered}{JetBrains Registrierung}{fig:jetbrainsRegister}
        \item Bewerben Sie sich für das Studentenprogramm von JetBrains (mit der Schul-E-Mail-Adresse).
        Dies bewirkt, dass ihr soeben erstellter Account kostenlos alle JetBrains Produkte für nicht kommerzielle Zwecke verwenden kann.
        \textbf{(Beantworten Sie alle Fragen des Webformulars wahrheitsgetreu)}
        \href{https://www.jetbrains.com/shop/eform/students}{JetBrains Students}
        (Wählen Sie die Sparte: \textbf{University Email Address})
    \end{enumerate}
\end{frame}

\subsubsection{PyCharm Pro Installation}\label{subsubsec:pycharmpro}
\begin{frame}[fragile]
    Für die Installation von PyCharm Pro öffnen Sie das Terminal und geben folgenden Befehl ein:\newline
    \textbf{sudo snap install pycharm-professional \texttt{--}classic}
\end{frame}

\subsubsection{RubyMine Installation}\label{subsubsec:RubyMine}
\begin{frame}[fragile]
    Für die Installation von RubyMine öffnen Sie das Terminal und geben folgenden Befehl ein:\newline
    \textbf{sudo snap install rubymine \texttt{--}classic}
\end{frame}