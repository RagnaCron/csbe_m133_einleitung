%%
%% Copyright (C) 2020 by Manuel Werder
%%
%% This work may be distributed and/or modified under the
%% conditions of the LaTeX Project Public License, either version 1.3
%% of this license or (at your option) any later version.
%% The latest version of this license is in
%%
%%     http://www.latex-project.org/lppl.txt
%%
%! Date = 15.05.20

\secStyle{Problemstellungen}
\label{problemstellung}
In diesem Kapitel werden diverse Problemstellungen betrachtet, die während des Projekts aufgetreten sind.

\subSecStyle{Aufsetzen von \LaTeX}
Die Form, die gewällt wurde für die Dokumentation, ist nicht Vorgabe für die LB,
aber sie bedarf doch einer schriftlichen Erwähnung, da es mein erstes Projekt ist,
bei der \LaTeX zur Anwendung kommt. Was aber nicht heisst das wir keine Dokumentation schreiben
müssen, ganz im Gegenteil, es gehört zur ÜK 335 eine entsprechende Dokumentation
zum Projekt zu erstellen.\newline\newline
Das Installieren von \LaTeX~ist nicht einfach, ich musste hierfür gleich zweimal antreten.
Beim zweiten Anlauf für die Installation hat es dann auch funktioniert.
Die Webseite \href{https://www.overleaf.com}{Overleaf} hat mir sehr geholfen mich mit
den ersten Konzepten vertraut zu machen, obwohl ich klar sagen muss, das ich sehr viel
Zeit dafür aufwenden musste, dass ich überhaupt jetzt eine Dokument habe an dem ich auch
gerne arbeite, da ich doch gewisse Ansprüche an Style und Aussehen habe. Es gibt aber
noch viel zu lernen und man kann auch noch sehr viel an diesem Dokument verbessern.

\subSubSecStyle{\LaTeX~und die Bibliographie}
Um ein Quellenverzeichnis zu machen, so wie ich das kenne, muss man so einiges Einstellen und
dann auch noch am richtigen Ort aufrufen. Das Ganze war nicht ganz einfach, aber zu guter Letzt
hat es dann auch funktioniert. Wo ich wirklich Schwierigkeiten hatte war das Hinzufügen vom
Tabellen- Abbild und Quellenverzeichnis in das Inhaltsverzeichnis.

\subSecStyle{Google-Play-Game-Services}
\label{googlePlayGame}
"Wo soll man da nun anfangen?"\newline\newline
Aller Anfang ist meistens schwer, nur dieser war auch noch unübersichtlich.
Wer die "Errungenschaften" wie auch die "Bestenliste" von Google Play Game Services verwenden will,
muss zuerst mal einen autorisierten Account bei Google Play Console haben. (Der Spass kostet einmalig 25\textdollar.)
Wenn man den Account hat, geht es weiter mit dem Erstellen eines Gameservice.
Dieser Gameservice muss dann spezifisch mit dem Packetnamen der man der Applikation entnimmt, verknüpft werden.
Was dann auch noch gemacht werden muss ist einen OAuth 2.0 Authentifizierung
durchzuführen mit seinem Projekt. Da man den Key benötigt, um die Applikation zu verknüpfen.
Und bis dahin hat man den Dienst noch nicht einmal verwendet. Was noch wichtig ist,
man muss auch noch entsprechende "Bestenlisten" und "Errungenschaften" erstellen,
sonst hat man das ja wirklich umsonst gemacht. Was auch noch wesentlich ist das man das App "Play Games"
von Google herunterlädt und einen Account erstellt.\newline\newline
Nun denn, wie man sieht ist das für ein erstes man nicht gerade einfach, besonders
dann nicht, wenn man in einem offiziellen Tutorial, dass einem pro gefühltem
Subkapitel in ein anders Tutorial führt, um dann irgend wann auch noch fertig zu werden mit dem Ersten Tutorial.
Ich habe die Verknüpfung aufgebaut und einen Account auf "Play Games" erstellt,
was jetzt noch fehlt ist das eigentliche Testen.

\subSecStyle{Game-Overlay}
\label{game-overlay}
"Wie legt man ein Overlay an für eine SurfaceView an?"\newline\newline
Wie auch in der reinen Java Entwicklung, ergibt sich in Android, dass es nicht ganz einfach
ist ein Spiel zu programmieren. Sich eine Übersicht über die Möglichkeiten zu verschaffen bedarf
Zeit und Ausdauer. Ein Overlay an zu wenden auf eine SurfaceView bedingt, dass man sie in eine
Gruppe einbindet und so dann eigentlich übergibt, um sie darstellen zu lassen. Bis ich verstanden
habe wie ich die Events entsprechend weiterleiten muss an die GameView Klasse hat es einen Moment
gebraucht, da ich zuerst den Ansatz verfolgt habe mit der Methode onMotionEvent zu arbeiten,
bis ich bemerkt habe, dass so das Game Play merklich darunter leidet. Ich habe mich dann eben
entschieden ein Overlay zu kreieren das Buttons hat, um so die Interaktion mit dem Benutzer zu
gewährleisten.

\subSecStyle{Sound Pool}
Als ich den ersten Effekt eingebunden hatte in Spiel, war ich schon sehr froh, es war
auch das erste Mal, dass ich effektiv Musik und Geräusche in einem Programm verwendet
habe. Aber beim Hinzufügen von multiplen Effekten, kam es dann zu merkwürdigen Exceptiones,
die zu kryptisch wahren, um sie einfach so zu verstehen, was aber wesentlich war, ist das man
es auch gehört hat. Ich habe dann Online ein wenig recherchiert und habe einen Artikel dazu
gefunden der mir auch noch bei weiteren Bugs und Problemen geholfen hat.

\subSecStyle{SettingsScreen}
Die Implementierung von SettingScreen kann man wirklich als trivial bezeichnen, aber bei der Einbindung
ist es dann zu einem Bug gekommen und ich musste da improvisieren. Leider konnte ich nicht für
alle drei Sounds den SoundPool verwenden und musste deswegen für den Hintergrundsound auf
den MediaPlayer ausweichen. Ich weiss nicht an was es gelegen hat, aber es hat immer nur die ersten
5 -10 Sekunden abgespielt bis er dann wieder von frone angefangen hat. Man kann diesen Fix aber nicht
als eine permanente Lösung anschauen, da es das Gesamtbild des Singetons zerstört, wenn man es liest.


